\documentclass{beamer}
\usetheme{Warsaw}

\title{Compromising privacy with the FLUSH+RELOAD attack}
\subtitle{CPSC 502.04 Presentation}

\author{Taylor Hornby \and John Aycock}

\begin{document}
% =============================================================================
% TITLE PAGE
% =============================================================================
    \frame{\titlepage}

% =============================================================================
% SIDE CHANNELS
% =============================================================================
\begin{frame}
    \frametitle{Side Channel Attacks}
    Your code might do the right thing, but it still leaks the key.
    \begin{itemize}
        \item Get RSA key from smartcard by looking at power use.
        \item Get RSA key by listening closely.
        \item Lots more.
    \end{itemize}
\end{frame}

% =============================================================================
% FLUSH+RELOAD
% =============================================================================

\begin{frame}
    \frametitle{\textsc{Flush+Reload}}
    \framesubtitle{Cache Architecture}
\end{frame}

\begin{frame}
    \frametitle{\textsc{Flush+Reload}}
    \framesubtitle{Sharing Physical Memory}
\end{frame}

\begin{frame}
    \frametitle{\textsc{Flush+Reload}}
    \framesubtitle{Spying on A Program}
\end{frame}

\begin{frame}
    \frametitle{\textsc{Flush+Reload}}
    \framesubtitle{Simple Example}
\end{frame}

\begin{frame}
    \frametitle{\textsc{Flush+Reload}}
    \framesubtitle{Breaking Crypto}
\end{frame}

\begin{frame}
    \frametitle{\textsc{Flush+Reload}}
    \framesubtitle{More than Crypto?}

    In theory every time a program branches it's leaking information about the
    input.

    \begin{itemize}
        \item Text editors?
        \item Email programs?
        \item Compression?
        \item GUI Apps? Can you see which buttons they are pressing? Mouse movements?
        \item Spell checking?
    \end{itemize}

    How bad is it? Let's try attacking things.
\end{frame}

\begin{frame}
    \frametitle{Outline}

    We found two attacks:
    \begin{itemize}
        \item TrueCrypt: Do you have a hidden volume?
        \item Links: Which of the top 100 Wikipedia pages did you visit?
    \end{itemize}
\end{frame}

% =============================================================================
% TrueCrypt Attack
% =============================================================================

\begin{frame}
    \frametitle{TrueCrypt Attack}
    \framesubtitle{Goal}

    \begin{itemize}
        \item TrueCrypt is full-disk encryption software.
        \item Hidden volumes.
        \item \textbf{Goal: Can we find out if the volume is hidden?}
        \item Your adversary will know to keep torturing you.
    \end{itemize}

    % TODO: put truecrypt screenshot/logo here, maybe hidden volume UI page?
\end{frame}

\begin{frame}
    \frametitle{TrueCrypt Attack}
    \framesubtitle{Implementation 1}

    Different \texttt{VolumeLayout} classes for different types of volumes.

    % TODO: put picture of TrueCrypt disassembly
\end{frame}

\begin{frame}
    \frametitle{TrueCrypt Attack}
    \framesubtitle{Implementation 2}

    We put probes on:

    \begin{enumerate}
        \item \texttt{\_start}
        \item \texttt{VolumeLayoutV2Normal::GetDataSize()}
    \end{enumerate}

    Watch the user mount a volume. If observed probe sequence ends with the
    \texttt{GetDataSize()} probe, the volume is normal. Otherwise, it's hidden.

\end{frame}

\begin{frame}
    \frametitle{TrueCrypt Attack}
    \framesubtitle{It Works}

    % TODO: introduce experiment systems

    \begin{itemize}
        \item Attack work on System 2, but,
        \item On System 1, it consistently gets it right over 80\% of the time.
    \end{itemize}

\end{frame}

% =============================================================================
% Links Attack
% =============================================================================

\begin{frame}
    \frametitle{Links Attack}
    \framesubtitle{Goal}

    % TODO: screenshot of lynx on wikipedia homepage

    \begin{itemize}
        \item Links (not Lynx or ELinks) is a command-line web browser.
        \item \textbf{Goal: We know you visited one of 100 Wikipedia pages, but
        which one?}
        \item We can probably tell which disease you are looking up.
        \item We used the Top 100 Wikipedia pages of 2013.
    \end{itemize}
\end{frame}

\begin{frame}
    \frametitle{Links Attack}
    \framesubtitle{Implementation 1}

    Put probes on the HTML parsing code. Trial-and-error to find the best
    locations.

    \begin{itemize}
        \item \texttt{parse\_html()}
        \item \texttt{html\_stack\_dup()}
        \item \texttt{html\_h()}
        \item \texttt{html\_span()}
    \end{itemize}

\end{frame}

\begin{frame}[fragile]
    \frametitle{Links Attack}
    \framesubtitle{Implementation 2}

    Spy on Links while it loads the page.


\begingroup
\fontsize{7pt}{7pt}\selectfont
\begin{verbatim}
HDRDSSSSDSDSDHDHDHDHDRDHDSSDSDSDSDHDHDHDRDRDRDHSRDRDRDRDRDSSDHDRDRDSDSDRDRDRDHDH
DRDHDRDRDRDSDSDSDHSDSRDSDRSSDHSSDSSDSDRDSDSDSDSDSDRDRDHSDSSSDSSRDSDRSSSDHDSDSDRS
DSDSDRDSDRDRDSSDSDSDSDRDSDSDSDSDSDSRDRDSDSDSRDRDRDRDRDSDSRDSDRDRDRDHSDRDRDRDRDRD
RDSDSDRDSDSDRDRDRDRDHSDRDRDSDSDSSDSSDRDRDSSDSSDSDSDRSDRDSSDSDSDSDSDRDSDRDSDSDSDS
RSDRDSHSDSDSDSRDRDSDSSDRDRDSHSDRDSDSRDHSDRDRDRDRSSDSDSDSDRDRDSDSDSDSDSRSRDHSSDSD
SDSDRDSDRDSDSDSSDRDSDRDSDSDSDSDSRDRDSDSDSRDSHSDRDSDSDSDSDRDSDSDSSDSDSDSSDSDSDRDS
RDRDRDRDHDSDRDRDRDRDHDRDRDRDRDRDRDRDRDRDRDRDRDRDRDRDRDRDRDRDRDRDRDRDSHSDRDSDRDHD
RDRHDHSDRDRDSDSSSDSDSDSDRDSDSDSDSDRDSDSDSDSDRDSDSDSDSDRDSDSDSDRSDSDSDRDHDHDSDSDS
DSDSDRDSHDSDSDSDSDSDRDSDSSDRDSDRDRHDSDSDRDSDRDHDSDSDSDSDHSDSSSDSDSSDHDSSSDSDHDHS
DSSDSDSDSSDSDSDSDSDSSSSDSDSSDSDSDSDSDSDSDSDSDSSDSDSDSDSDSSDSDSDSDSDSDSSDSDSDSDSD
SDSDSSDSDSDSDSDSDSDSDSDSDSDSSDSSSDSSDSDSSDSDSDSDSDSDSDSDSDSDSDSDSSDSDSSDSDSDSDSD
SSDSDSSDSSDSDSSDSSDSDSDSDSSDSDSDSDSSDSDSDSDSSDSDSDSDSSDSDSDSDSSDSDSDSSDSDSDSDSDS
SDSSDSDSDSDSSDSDSDSDSDSDSSDSDSSSDSDSDSDSDSDSDSDSDSHSSDSSSDSSDSDSSSDSDSDSDSDSDSDS
DSSDSDSSDSSDSDHDSSDSDHDHDSSDSSDSDSDSSDSDSDSDSDSDSDSDSDRDSSDSDSDSDSDSDSDSDSDSDHSD
SDSDSDSDHSDSDSDSHSDSDSDHSSDSDSDSDSSDSDSDSDSDSDSDHDHDSDSDSDSDSDSDSDSDSDSDSDSDSDSD
SHDHDSDSDSDSDSDHDSDSDSDSDSDSDSDSDSDSDSDSDSDSDSDSDSDSSDSDSSDHDSDSDHSDRDHDSDSDSSDS
DSDSDSDSDSSHSDSSDSDSDSDSDSDSDSDSDSDSDSSDSDSDSDSDSDSDSSDSDSDSDSDSDSDSDSDSDSDHDSDS
DSDSDSDSDHDSSDSDSDSDSDSDSDSDSDSDSDSSDSDSDSDSDSDSDHDSDSDSDSDSDSDHDSDSDSDSDSDSDSDS
SDHSDHDSSDSDSDSDSDSDSDSDSDSDSDSDSDHDSDSDSDSDSSDSDSSDSDSDSDSSDSDSHDSDSDSDSDSDSSDS
DSSDSDSSDSDSDSDSDSDSDSDSDSDSDSDSDSDSDSDSDSDSDSDSDSDSDSSDSDSDSSDSSDSDSDSDSDSDHSDS
DHSHDSSSDSDSDSDRDSDSSDSDSDSDSDSDSDSSDSDSDSDSDHDSSHDSDRDSDRDSSRDRDHDSDSRDRDRDSSDR
DSDSDSRDRDRDSDSRDRDSDSDRDRDSDSDSRDRDSDSRDRDRDRDRDSDSSDRDRDSDSRDRDSDSRDRDRDSDSDSD
RDRDSDSDSRDRDSDSSDRDSDSRDRDSDSRDRDRDRDRDSDSDSRDRDSDSRDRSDSDSRDRDRDSDSDRDRDRDSDSD
SRDRDSDSSDRDRDSDSDSRDRDRDSDSSDRDRDRDRDRDRDRDRDRHDSDSDRDSDSSDSDSDSDSDSDSDSDRDSDSD
SDSDSDSDRHDSDSDRDRDRDRDRDRDSDSSDSDHSDSSDSDSDSDSDSDSSDSDSDSDSDSHSDSDHDSDSDSSDSDHD
SDSDSDSSDSDSDSDHSDSDSDSSDSDSDSDSDSDSSDHSDHDSDSDSDSDHSSDSHDSDRDHDSDHDHSDSDSDHDSDS
SDSDSDSDSDSDSDSDSDSDSDSDSDSDSDSDSHDSDSDSDSSDSDSDSDSDSDSDSDSDSDSDSSDHDSSDSDSDSDSD
SDSRDSDSSDSDSDSDSHDHDRDHDHDHDHDHDSDSDSDSDS...
\end{verbatim}
\endgroup

\end{frame}

\begin{frame}
    \frametitle{Links Attack}
    \framesubtitle{Implementation 3}

    Phases of attack:

    \begin{enumerate}
        \item Training: Just save 5 to 10 samples of each page.
        \item Spy on the victim: Record the victim visiting a page once.
        \item Recovery: Find the sample with the smallest Levenshtein distance.
    \end{enumerate}
\end{frame}

\begin{frame}
    \frametitle{Links Attack}
    \framesubtitle{Levenshtein Distances}

    % TODO: the youtube LD graphic

\end{frame}

\begin{frame}
    \frametitle{Links Attack}
    \framesubtitle{Demo}
    Demo!
\end{frame}

\begin{frame}
    \frametitle{Links Attack}
    \framesubtitle{It Works}

    % TODO: introduce experiment systems

    \begin{itemize}
        \item Works very well on both System 1 and System 2.
        \item 10 samples of each page: Gets it right about 90\% of the time.
        \item 5 samples: about 75\%.
    \end{itemize}

    % TODO: chart showing most pages are reliably identified

\end{frame}

% =============================================================================
% Conclusion
% =============================================================================

\begin{frame}
    \frametitle{Conclusion}

    \begin{itemize}
        \item These are boring attacks.
        \begin{itemize}
            \item TrueCrypt: 1 bit.
            \item Links: $\log_2(100) = 6.6$ bits.
        \end{itemize}

        \item Can we automate attack discovery?
    \end{itemize}
\end{frame}

\begin{frame}
    Thanks for listening! Questions?
\end{frame}

\end{document}
