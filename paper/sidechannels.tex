
% !TEX TS-program = pdflatex
% !TEX encoding = UTF-8 Unicode

% This is a simple template for a LaTeX document using the "article" class.
% See "book", "report", "letter" for other types of document.

\documentclass[letterpaper,10pt]{article} % use larger type; default would be 10pt

\usepackage[utf8]{inputenc} % set input encoding (not needed with XeLaTeX)

%%% Examples of Article customizations
% These packages are optional, depending whether you want the features they provide.
% See the LaTeX Companion or other references for full information.

%%% PAGE DIMENSIONS
\usepackage{geometry} % to change the page dimensions
% \geometry{margins=2in} % for example, change the margins to 2 inches all round
% \geometry{landscape} % set up the page for landscape
%   read geometry.pdf for detailed page layout information


\usepackage{usenix}

%%% PACKAGES
\usepackage{indentfirst}
\usepackage{booktabs} % for much better looking tables
\usepackage{array} % for better arrays (eg matrices) in maths
\usepackage{paralist} % very flexible & customisable lists (eg.enumerate/itemize, etc.)
\usepackage{verbatim} % adds environment for commenting out blocks of text & for better verbatim
\usepackage{subfig} % make it possible to include more than one captioned figure/table in a single float
\usepackage{amsthm}
\usepackage{amsmath}
\usepackage{nccmath} 
\usepackage{algorithmic}
\usepackage{bm}
\usepackage[numbers]{natbib}
\usepackage{graphicx} % support the \includegraphics command and options

%%% The "real" document content comes below...

\title{Side Channels in Non-Cryptographic Applications are Realistic Threats}
\author{Hornby, Taylor\\
    \texttt{antispam@example.org}
    \and
    Aycock, John\\
    \texttt{antispam@example.org}
}
%\date{} % Activate to display a given date or no date (if empty),
         % otherwise the current date is printed

\newtheorem{theorem}{Theorem}
\newtheorem{lemma}{Lemma}

\usepackage[parfill]{parskip}

\begin{document}
\maketitle

\begin{abstract}
\end{abstract}

\section{Introduction}

\section{Previous Work}

\section{Attacks}

\subsection{TrueCrypt}

\subsubsection{Recovering Cipher Choice}

TrueCrypt supports different ciphers. Because the cipher choice is not stored in
in the metadata, it is impossible to find out which cipher is in use without
knowing the password. To decrypt a volume, TrueCrypt itself must try all
possible combinations of ciphers and hashes. Therefore, an attacker must do the
same.

The attacker is interested in knowing which cipher is in use, since it will
speed up their dictionary attack on the password. We applied FLUSH+RELOAD to
determine whether the Serpent or Twofish ciphers were in use when a volume is
mounted.

To do so, we placed probes on the \texttt{serpent\_decrypt} and
\texttt{twofish\_decrypt} methods. By observing the output, it is obvious
whether the cipher is Serpent or Twofish.

TODO: actual experiment, more description of how it works.

\subsection{Vulnerable Program 2}
\subsection{Vulnerable Program 3}

\section{Defense}

\subsection{Defense Mechanism 1}
\subsection{Defense Mechanism 2}
\subsection{Defense Mechanism 3}

\section{Related Work}

\section{Conclusion}

Test a citation \citep{Pappas2012SmashingGadgets}.

\bibliographystyle{acm}
\bibliography{sidechannels}

\end{document}


